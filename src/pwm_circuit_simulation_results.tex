\section{PWM Circuit Simulation Results}
\subsection{PWM Generator Simulation Results}
% The simulation results presented should validate that the design meets the specification.
% Include diagrams and figures to support your presentation of results.
% In addition to presenting the simulation results include an answer to the following questions. Make sure that the answers to the questions are clearly indicated in your report.
\subsubsection{Explain the design basis for the choice of op-amp power supply voltages, +Vs and -Vs? What is the minimum value which could be used for these?}

\subsubsection{What is the power consumption of the PWM circuit? What is this power used for?}
\begin{table}[h]
	\centering
	\caption{Power Consumption}
	\begin{tabular}{ll}
		\toprule
		Component&Power Consumed \si{\watt}\\
		\cmidrule(r){1-1}\cmidrule(l){2-2}
		PWM Power Supply & $\num{-110.49e-3}+\num{-171.74e-3}=\num{-0.28223}$\\
		Converter Power Supply & \num{-5.0605}\\
		Converter Output Voltage & \num{5.0701}\\
		\bottomrule
	\end{tabular}
	\label{tab:pwm power}
\end{table}
The power consumption of the PWM circuit is as given in \cref{tab:pwm power}
\subsubsection{What is the effect of the PWM circuit power consumption on the converter efficiency?}
%Quantify your answer.

\subsection{PWM Generator \& Buck Converter Results}
% The simulation results presented should validate that the design meets the specification.
% Include diagrams and figures to support your presentation of results.

% In addition to presenting the simulation results include an answer to the following questions. Make sure that the answers to the questions are clearly indicated in your report.
\subsubsection{What is the relationship between output voltage and reference voltage?}
% You could investigate this with a graph of output voltage vs. reference voltage.
\subsubsection{Could the same PWM circuit be used to drive the converter at 10 times your design frequency? What might limit the maximum frequency of operation of this circuit?}

