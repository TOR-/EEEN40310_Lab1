\section{Initial Converter Test Results}

\begin{table}[h]
	\centering
	\caption{Inductor Design, build and test}
	\begin{tabular}{lcl}
		\toprule
		Quantity&Value&Comments\\
		\cmidrule(r){1-1}\cmidrule(lr){2-2}\cmidrule(l){3-3}
		Target Inductance Value & \SI{0}{\henry} & \\
		Required number of turns & 0 & \\
		Calculated Coil Resistance & \SI{0}{\ohm} & \\
		Measured Inductance & \SI{0}{\henry} & \\
		Actual number of turns & 0 & \\
		Measured resistance (low frequency) & \SI{0}{\ohm} & \\
		Measured resistance (switching frequency) & \SI{0}{\ohm} & \\
		\bottomrule
	\end{tabular}
\end{table}

\subsection{Initial Converter Test Results (driven by signal generator)}

% The basic operation of the Buck converter should be verified by using the signal generator to provide the drive voltages for the MOSFET. Results/oscilloscope images should be included which show the operation of the converter. The results from the tests should be compared to the simulation results. Note the earlier simulations results may have to be changed to match the actual test conditions.

% In addition to presenting the test results include an answer to the following questions: In your report clearly indicate the question and the answer.

\subsubsection{How has the use of the signal generator (to drive the MOSFET) limited the tests which can be performed?}

\subsubsection{Do the test results and the simulation results agree?}
% for this question avoid using inexact terms like “roughly the same” or saying they are the same when they are not. Ideally you should quantify any differences and try to explain them

