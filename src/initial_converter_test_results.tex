\section{Initial Converter Test Results}

\begin{table}[h]
	\centering
	\caption{Inductor Design, build and test}
	\begin{tabular}{lcl}
		\toprule
		Quantity&Value&Comments\\
		\cmidrule(r){1-1}\cmidrule(lr){2-2}\cmidrule(l){3-3}
		Target Inductance Value & \SI{72}{\micro\henry} & Minimum inductance as previously calculated plus safety factor \\
		Required number of turns & 9 & $N=ceiling\left\lbrace \sqrt{\frac{L}{A_L}}\right\rbrace$ \\
		Measured Inductance & \SI{83.2252}{\micro\henry} & \\
		Actual number of turns & 10 & \\
		Measured resistance (low frequency) & \SI{0.01937}{\ohm} & \\
		Measured resistance (switching frequency) & \SI{0.02}{\ohm} & \\
		\bottomrule
	\end{tabular}
\end{table}

\subsection{Initial Converter Test Results (driven by signal generator)}

% The basic operation of the Buck converter should be verified by using the signal generator to provide the drive voltages for the MOSFET. Results/oscilloscope images should be included which show the operation of the converter. The results from the tests should be compared to the simulation results. Note the earlier simulations results may have to be changed to match the actual test conditions.

% In addition to presenting the test results include an answer to the following questions: In your report clearly indicate the question and the answer.


\begin{figure}
	\centering
	\includegraphics[width=\linewidth]{"img/Converter Output"}
	\caption{Converter output with ripple}
	\label{fig:converter op}
\end{figure}


The output voltage for a load of \SI{10}{\ohm} is given in \cref{fig:converter op}. The converter was also tested with loads of \SI{4.7}{\ohm} and \SI{1}{\ohm}. We were unable to provide enough current for the circuit to work with the \SI{1}{\ohm} load.

The output voltages were \SI{5.12}{\volt} for the \SI{10}{\ohm} load (current \SI{0.512}{\ampere}) and \SI{4.73}{\volt} for the \SI{4.7}{\ohm} load (current \SI{1}{\ampere}).

The power input to the circuit was measured for the \SI{4.7}{\ohm} max current load by noting the current drawn from the power supply. This gave $P_{in}=\SI{6.5}{\watt}$, giving an efficiency of 74\%.

\subsubsection{How has the use of the signal generator (to drive the MOSFET) limited the tests which can be performed?}
It can't emulate the power converter reacting to changes in the input voltage. It also can't vary the duty cycle, being fixed at 0.5.
\subsubsection{Do the test results and the simulation results agree?}
% for this question avoid using inexact terms like “roughly the same” or saying they are the same when they are not. Ideally you should quantify any differences and try to explain them
The following statements apply to the \SI{4.7}{\ohm} load circuit.

The efficiencies are different, with the physical circuit being 12 percentage points more efficient than the simulated circuit. I would have expected this to be the opposite, with the simulation being more efficient than reality. Perhaps the esrs are overestimated or the PMOS model is too lossy.

The simulated output voltage was 4.69 V whereas the tested voltage was measured at 4.73 V. These are rather close.
